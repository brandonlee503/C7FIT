\subsection{Brandon Lee}
\subsubsection{What technical information did you learn?}
From this experience, I was able to hone my mobile development skills, specifically in the realm of iOS. I was able to read up on various Cocoa frameworks and tools such as UICollectionViews and eBay's Browse APIs. Utilizing these technical skills, I built various screens including the Store screen and Profile screen. Additionally, I learned Google's Firebase and ways to integrate their services into an iOS project. Because of our client's expressive desire to keep as natively as possible, I developed knowledge in doing things 'the hard way' in iOS including laying out UI with Autolayout and NSAnchors, making network calls with URLSession rather than Alamofire and dealing with JSON natively rather than with SwiftyJSON.

\subsubsection{What non-technical information did you learn?}
I learned a lot about myself and my interests. I find that iOS development is definitely something that I want to remain in as a career field. In terms of work ethic, I find that as with anything I spend the majority of my time with, I occasionally need to take a break from it in order to renew my interest from time to time. All in all, though, I find that I found the field I wish to be in.

\subsubsection{What have you learned about project work?}
I found that project work can be difficult at times when the team is fragmented remotely. Communication is the main issue here and it is absolutely critical that there is some form of scheduled group meeting. This is to ensure that progress is steadily being maintained throughout the terms.


\subsubsection{What have you learned about project management?}

Project management is difficult. There needs to be some form of control between the team leader and the other members. It can be hard to do this when we're all peers as well. I found that meeting face to face on frequent occasions can help in alleviating this struggle and keeping development and progress on track. I also learned that I'd rather be a worker/follower rather than a leader.

\subsubsection{If you could do it all over, what would you do differently?}

If I could do it all over, I'd tell myself to start earlier, such as the Fall term as we spent a lot of time simply writing rather than developing. Additionally, I think I would have made weekly group meetings more mandatory so that all the group members would have the opportunity to touch bases more on a routine basis throughout the year.

\subsection{Rutger Farry}
\subsubsection{What technical information did you learn?}
This year I was provided with the opportunity to refine my iOS development skills and expose myself to various programming paradigms in iOS. Between work, capstone, freelancing, and personal projects, I was simultaneously working on at least three iOS projects during capstone, which actually turned out to be pretty great. Since I had my hands in so many projects, each with a different way of doing things, I was able to take my successes and failures and apply what I learned to C7FIT. 

I learned that developing entirely in code is really good for large teams, but oftentimes using Xcode's Interface Builder is a good option for quickly developing complex interfaces, and shouldn't be left out of the iOS developer's toolbox for small projects.

Since we forced ourselves to develop the interface entirely in code though, I discovered that there are actually some great new ways to design with code in iOS that are much less painful than previous methods. UIStackView, which is basically Flexbox for Cocoa, became my best friend in designing hierarchical interfaces.

\subsubsection{What non-technical information did you learn?}
One of the most important things I learned while working on my capstone project was how to organize my work over time. For most of my high school and university "career", I mostly just did projects the day before they were due, no matter the size. This year, with capstone, organizing a cycling race, and taking a lot of Computer Science classes, I learned that life is a lot less stressful if you prepare for things beforehand.

\subsubsection{What have you learned about project work?}
Once again, staying organized is very important. Splitting up work into small chunks and having frequent check-ins are helpful. We siloed off the app into large chunks and for the most part, didn't mess with each other's code. This helped us avoid merge conflicts but probably resulted in less shared code.

\subsubsection{What have you learned about project management?}
Splitting a project into sub-tasks is sometimes hard before starting work, but an effort should be taken to do this as soon as possible early on in the project. Also laying out a solid foundation and defining idioms is important for maintaining consistency.

\subsubsection{What have you learned about working in teams?}
I've worked in teams a lot in the past so this was not very new. I would say that constant communication is important. Also making regular contributions, even if you are busy and can only make small contributions, shows that you are "in" with the project and are concerned about its success.

\subsubsection{If you could do it all over, what would you do differently?}
I would've done more research into the new tools and services we were going to use – this would've helped us avoid the awkward mid-year transition away from MINDBODY. If we had known we couldn't use MINDBODY sooner we probably would've been able to plan better. I would also have made an effort to reach out to our clients at eBay and Club Seven Fitness more since that would give us a better idea of who our target audience is. When engineers get too out of touch with their clients, their product may make sense to them but can be absurd to normal users. While I don't think this is the case with our application, it would've still been nice to talk more.

Additionally, although the first term was very writing-intensive, I would've like to squeeze a little time into starting development in the first term. It seems like a lot of this class was based on the naive assumption that building a product will be much smoother if extremely detailed plans are written out beforehand. The truth is that many things can only be known after starting to get your hands dirty, and while planning is certainly essential, a lot of the details we were expected to write about were impossible to know before starting to code. Therefore I would've told myself to start development earlier.

\subsection{Michael Lee}

\subsubsection{What technical information did you learn?}
First and foremost, I learned about iOS development and with it comes all of the technologies and development kits that are native to Apple its iPhone products. This primarily includes the Xcode development environment, the Swift language, and HealthKit and MapKit SDKs. For our project, we chose to develop the interface purely in code which was also a learning hurdle for me. Normally for beginners, they start off using Apple’s drag and drop interface builders which makes it much easier, but because of the eBay team and our design decisions we opted for the more difficult route: However, from this, I’ve learned the best design practices for iOS app development from the eBay team and mine.

\subsubsection{What non-technical information did you learn?}
Working long term within the group I’ve learned the importance of weekly meetings that keep the whole team on track. I’m sure in a working environment when everyone is in the same environment and not split between school teams are more on the same page. This has shown me the importance of communication between teams and how helpful it is to update teammates on your status. 

\subsubsection{What have you learned about project work?}
I’ve learned that not only is good and thorough planning key to a successful project, not everything will always go to plan. For this project we spent a large amount of time planning, a whole term, simply writing documents about what we wanted to do. The whole term almost went to waste as we found out we weren’t able to afford the API that our project was centered around. This showed me the importance of flexibility as we needed to 

\subsubsection{What have you learned about project management?}
I’ve learned that to be successful a team needs to be able to respond to major changes quickly and efficiently. Planning is not just a one-time thing, the process of planning begins before the project and continues throughout the development process. As circumstances change you can’t just stick to an archaic outdated plan, but instead, you must adapt and change to fit the new circumstances.

\subsubsection{What have you learned about working in teams?}
As I’ve stated before working within teams on such a large project takes a lot of communication and accountability. It’s easy to have tunnel vision when you focus on your portion of the project and it definitely helps to have group mates review and debug your code for the sake of functionality and understandability. As a group member, you need to remain accountable for your portion of the code, as not only does the success of a project as a whole depend on your work, other group mates could depend on your code for their portions. When a group member doesn’t complete their portion promptly it can lead to a general mistrust or doubt that one will fulfill their duties, but this can be solved conversely with good communication. Sometimes confrontation is necessary in order to get things done.

\subsubsection{If you could do it all over, what would you do differently?}
If I could restart, I would interact more with my client and the software developers of eBay. The developers were a resource that I should've taken more advantage of, and seeing all of the cool projects that were created at the OSU Engineering Expo, I would've liked to try implementing things more innovative or advanced even if they might not be entirely successful. \\

I would tell myself all the normal things like start earlier and work continuously throughout the term, but overall I didn't lag too behind. It would have been better for me to start earlier as I had much more to catch up on than my group mates.