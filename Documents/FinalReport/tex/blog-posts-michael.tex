\lstdefinelanguage{swift}
{
  morekeywords={
    func,if,then,else,for,in,while,do,switch,case,default,where,break,continue,fallthrough,return,
    typealias,struct,class,enum,protocol,var,func,let,get,set,willSet,didSet,inout,init,deinit,extension,
    subscript,prefix,operator,infix,postfix,precedence,associativity,left,right,none,convenience,dynamic,
    final,lazy,mutating,nonmutating,optional,override,required,static,unowned,safe,weak,internal,
    private,public,is,as,self,unsafe,dynamicType,true,false,nil,Type,Protocol,
  },
  morecomment=[l]{//}, % l is for line comment
  morecomment=[s]{/*}{*/}, % s is for start and end delimiter
  morestring=[b]" % defines that strings are enclosed in double quotes
}

\definecolor{keyword}{HTML}{BA2CA3}
\definecolor{string}{HTML}{D12F1B}
\definecolor{comment}{HTML}{008400}

\lstset{
  language=swift,
  basicstyle=\ttfamily,
  showstringspaces=false, % lets spaces in strings appear as real spaces
  columns=fixed,
  keepspaces=true,
  keywordstyle=\color{keyword},
  stringstyle=\color{string},
  commentstyle=\color{comment},
}

\subsection{Michael Lee}

\subsubsection{October 14, 2016}\label{section}
The progress of this week has kept up with the pace of this class. Last
week we had the first phone meeting with our client where we hashed out
the basics details of our project. We had a group meeting after that to
work out the details of our problem statement.

Overall the communication between our group and client were good. It was
slightly difficult because they are remote, however we had good email
communication and we set up a slack to help. There were some
miscommunications due to online communication when getting our
signatures, however because our client responded promptly we were able
to complete the assignment on time.

Our group plans to go up to Portland to meet with our client next week
Wednesday so we can talk about the app in detail and meet the team that
were going to work with. As I haven't really worked in depth with IOS
programming, I will need to catch up to my teammates and so I plan on
learning some IOS in the coming week.

\subsubsection{October 21, 2016}\label{section}
Our team hasn't had too many problems, as we've already met with our
client twice. Our biggest problem would be working on the next
requirements document and determining what is feasible in the time we
have, as from the meetings we've had the project seems quite large.

On Wednesday we went up to Portland to meet with our client eBay. They
showed us around their office and we were able to meet the team members
we will be working with. During the meeting we discussed what
technologies we will be using on our app and the look of each of the
menus. We received our edited Problem statement Thursday night and we'll
be revising that later this week.

For the future, we need to iron out a more official requirements
document. Even though we have a preliminary version, we need to be clear
on the particular functions. To do this we'll be using the rough
wireframes that Luther provided us and our initial requirements
document.

\subsubsection{October 28, 2016}\label{section}
This week we were able to complete our requirements doc on time, but our
team was a little strapped for time. Most of our team was busy
throughout the week, so most of the requirements doc was done close to
the deadline.

This process for this week was mainly just the creation of the
requirements document. We did get a list of functional requirements from
our client, however it was too late to be implemented into the current
iteration of our requirements document. As a team we met to talk about
some of the higher level design of the app, which will be useful for
when we create the design document.

For the coming week we need to finalize our requirements document and
continue to work on the higher level design of our application.

\subsubsection{November 4, 2016}\label{section}
This week we didn't do much as a group and I had two midterms to deal
with, so the progress on capstone was stagnated.

For next week we need to get together as a group and plan our technology
review and Implementation plan

\subsubsection{November 11, 2016}\label{section}
Personally, I'm more inexperienced than my other group members in IOS
programming, so the implementation document requires more research for
me to complete my section.

Our team met up and we discussed the implementation document, and have
begun researching the different technologies we can use to complete our
project.

For the next week I need to finish researching and writing the document
so that it is complete by Monday.

\subsubsection{November 18, 2016}\label{section}
\paragraph{Progress}\label{progress}
This week we completed the technology review early in the week. We also
met up to make some preliminary plans for our Design document.
\paragraph{Problems}\label{problems}
With midterms from other classes, it's hard to coordinate with group
members, and it's unclear what we need to do for the design document
\paragraph{Plans}\label{plans}
Continue to make progress on the design document

\subsubsection{November 25, 2016}\label{section}
\paragraph{Progress}\label{progress}
We haven't made any progress this week due to the holidays.
\paragraph{Problems}\label{problems}
We didn't make any progress this week.
\paragraph{Plans}\label{plans}
We plan to work on the design document next week.

\subsubsection{December 2, 2016}\label{section}
\paragraph{Progress}\label{progress}
We completed our design document and turned it in without a signature,
as we completed it late Thursday we didn't have time to get our client
to sign it.
\paragraph{Problems}\label{problems}
The expectations of the design document were very unclear going into it
so it took much longer to complete than expected.
\paragraph{Plan}\label{plan}
Start working on the Progress Report

\subsubsection{January 13, 2017}\label{section}
Winter break finished, but only Brandon started working on the project, our group as a whole has not made much progress.

\subsubsection{January 20, 2017}\label{section}
Our client has informed us that they cannot afford the MindBody API which means the direction of our project needs to be changed

\subsubsection{January 27, 2017}\label{section}
This week was spent working on the easier schedule screen that has been modified to be a simple weblink to the gym's website.

\subsubsection{February 3, 2017}\label{section}
The schedule screen is finished and I've begun looking into the healthkit api and how we can use it to query a user's pre-existing healthkit data.

\subsubsection{February 10, 2017}\label{section}
I've begun work on the location tracking portion of the application with the mapkit api. It is difficult to get the mapkit to record accurate data and display it as a line for the user to see as they run.
\subsubsection{February 17, 2017}\label{section}
Mapkit has been implemented roughly, as it works to track the user's location. There needs to be some fine tuning and bug fixes done so that the location is tracked when the user isn't moving or moves very little.

\subsubsection{February 24, 2017}\label{section}
The map portion of the screen is finished and it allows the user to save their runs to the offline database firebase.

\subsubsection{March 3, 2017}\label{section}
improved the look of the run lookup detail view, so the user can see a small snapshot of their run before they click to load the full details.

\subsubsection{March 10, 2017}\label{section}
Most of this week has been spent developing the less important features of the activity tools screen like the stopwatch and countdown timer. They are relatively simple features but must be implemented based on the requirements doc

\subsubsection{March 17, 2017}\label{section}
This week was spent trying to develop a heart rate monitor from the data that can be recorded using the flashlight and the camera application on the iPhone. Currently, the phone can record the data, but analyzing it has proven to be a harder task than initially thought and the results aren't very accurate. This wasn't in the requirements document, though, so it may not be implemented.

\subsubsection{March 24, 2017}\label{section}
Progress: This past week I spent most of the time refining portions of
my code in preparation for the final clone that is coming on May 1st.
The implementation of the heart rate screen is completed, but visually
it is quite lacking. I added some additional modal views that displays a
more informative message to the user based on the heart rate that they
enter.

Problems: The finishing touches to our application will be primarily
making it look better, but the challenge with this is we weren't given
too many image assets and that will impact the final look of our
project. Additionally, because we're developing primarily in code rather
than using interface builder or storyboards making the views visually
appealing takes much longer to do than normal. There is a time crunch
coming up with the deadline approaching and so we need to focus to make
the project as visually presentable as possible. We also want to tie in
the healthkit data source with the firebase data that we've created, but
the functions have all been written we just need to link up to a single
source of truth

Plans: Next week we plan to go on a coding grind to make our project
look as good as we can make it. Development will continue after May 1st,
but we want to make it look as good as we can. The functionality is done
so thats all we have left to focus on.

\subsubsection{April 28, 2017}\label{section}
Progress: This week we spend most of it refining all of the parts of the
application to meet the specific requirements that we've outlined
previously in our requirements document. This was done in preparation
for the code freeze that we've been expecting on Monday. Our progress
was good overall, and I would expect that we will finish on time.

Problems: The problem we've run into is that our requirements weren't
evolving to an extent. We made some requirements so specific, that when
we realized the application wouldn't use that anymore we couldn't change
it. For the sake of the code freeze we made our application match the
requirements document as close as possible.

Plans: We plan on going to meet with our client Monday to show them the
current state of our application and see if they want to make any
changes post code freeze for the sake of uploading to the apple iOS
store.

\subsubsection{May 5, 2017}\label{section}
The code freeze has been implemented so development is relatively halted. There isn't much to do before expo anymore, and most of my work is centered on other classes as school gets busier

\subsubsection{May 19, 2017}\label{section}
Progress: We finished the midterm progress report early this week. We
completed expo this weekend, so a major milestone is finished.

Problems: A lot of things to do before expo happens, along with the
mental preparation that comes with it. Standing at expo for hours was
not pleasant.

Plans: Meet-up with the client and finalize everything before the end of
the term. Complete the remaining required documents for our class.

\paragraph{If you could redo the project from fall term, what would you
tell
yourself?}\label{if-you-could-redo-the-project-from-fall-term-what-would-you-tell-yourself}

If I could restart, I would interact more with my client and the
software developers of eBay. The developers were a resource that I
should've taken more advantage of, and seeing all of the cool projects
that were created at expo I would've liked to try implementing things
more innovative or advanced.

I would tell myself all the normal things like start earlier, and work
continuously throughout the term, but overall I didn't lag too behind.
It would have been better for me to start earlier as I had much more to
catch up on than my group mates.

\paragraph{What's the biggest skill you've
learned?}\label{whats-the-biggest-skill-youve-learned}

The greatest skill I learned would be iOS development. I got experience
with Apple's frameworks and SDKs like healthKit and mapKit, in addition
to learning iOS. I would liken the experience to learning web
development, and there is still much to learn in this aspect.

\paragraph{What skills do you see yourself using in the
future?}\label{what-skills-do-you-see-yourself-using-in-the-future}

I definitely see myself bringing this skill with me in future jobs, as
phones become more and more prevalent. All of the experience with
Apple's technologies will also help me if I want to develop on any of
their devices in the future.

\paragraph{What did you like about the project, and what did you
not?}\label{what-did-you-like-about-the-project-and-what-did-you-not}

I liked that it gave us practical experience in a field that does seem
to be growing: phones are becoming more prevalent and thus it's useful
to learn. On the other hand, I was not particularly interested in
developing user applications. I do wish that our project had more
research applications so it was more interesting, but it came with
practical applications.

\paragraph{What did you learn from your
teammates?}\label{what-did-you-learn-from-your-teammates}

I learned a lot about the best practices for developing in iOS, as both
of them were experienced in that area being from OSU's app club.
Communication and even though we are CS students, meeting up in person
is very helpful. This is a little difficult as were all in school and
have different schedules, but useful to note in the future workplace.

\paragraph{If you were the client for this project, would you be satisfied
with the work
done?}\label{if-you-were-the-client-for-this-project-would-you-be-satisfied-with-the-work-done}

I would be satisfied as we completed all of the requirements initially
outlined for us by our client. In addition to just the requirements the
application looks fairly professional especially given the assets that
we were given. Our app looks just as good if not better than other
comparable gym applications, and it looks more modern than their current
website. Our application combines fitness tools and gym specific
information together which many other applications do not.

\paragraph{If your project were to be continued next year, what do you
think needs to be working
on?}\label{if-your-project-were-to-be-continued-next-year-what-do-you-think-needs-to-be-working-on}

We would need to work on some documentation, but the requirements our
client as outlined for us were all completed. The app could use
development on additional features that would make it a better gym app
like integration with watchKit for the apple watch or integration of
some external api like the MindBody Api that the gym couldn't afford
previously.
